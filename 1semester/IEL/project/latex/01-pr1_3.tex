\begin{figure}[!ht]
\begin{center}
    \begin{center}
\begin{circuitikz}[]
\draw
  to[dcvsource, v<=$U$] ++(0, 2)
  to[short] ++(2, 0)
  to[resistor, l=$R_i$] ++(0, -2)
  to[short] ++(-2, 0)
;\end{circuitikz}
\end{center}
    Zjednodušíme sériovo zapojené $R_B$, $R_7$ a $R_C$, $R_6$ na $R_{B7}$  $R_{C6}$
    \[
        R_{B7} = R_B + R_7 = \frac{244800}{2371} + 330 = \frac{1027230}{2371} = \SI{433,24757}{\ohm}
    \]
    \[
        R_{C6} = R_C + R_6 = \frac{322500}{2371} + 800 = \frac{2219300}{2371} = \SI{936,01856}{\ohm}
    \]
    \newline
    \begin{center}
\begin{circuitikz}[]
\draw
  to[dcvsource, v<=$U$] ++(0, 3)
  to[short, -*] ++(1, 0)
  (1, 2) to[short, *-*] ++(0, 2)
  to[resistor, l_=$R_1$, v^>=$U_{R1}$] ++(2, 0)
  node[circ, label=$A$]{}
  to[resistor, l=$R_{45}$, -*] ++(2, 0)
  to[short,-*] ++(0, -2)
  to[resistor, l=$R_6$] ++(-2, 0)
  node[circ,label=below:{$B$}]{}
  to [open, v<=$U_e$] ++(0, 2)
  (3, 2) to[resistor, l_=$R_2$, v^<=$U_{R2}$] ++(-2, 0)
  (0, 0) to[short] ++(5, 0)
  to[short] ++(0, 2)
  (2, 3) node[/tikz/circuitikz/bipoles/length=1cm, circulator]{}
;\end{circuitikz}
\end{center}

    Pokračujeme zjednodušením paralerneho zapojenia $R_{B7}$ a $R_{C6}$, zjednodušíme ich na $R_{B7C6}$
    \[
        R_{B7C6} = \frac{R_{B7} * R_{C6}}{R_{B7} + R_{C6}} = 
        \frac{\frac{1027230}{2371} * \frac{2219300}{2371}}{\frac{1027230}{2371} + \frac{2219300}{2371}} = 
        \frac{227973153900}{769752263} = \SI{296,16432}{\ohm}
    \]
    \begin{center}
\begin{circuitikz}[]
\draw
  to[dcvsource, v_>=$U_{12}$] ++(0, -2)
  (0, 0) to[short, i_=$I$] ++(2, 0)
  to[resistor, l=$R_{ekv}$] ++(0, -2)
  to[short] ++(-2, 0)
;\end{circuitikz}
\end{center}
    Zostali nám už len sériovo zapojené odpory $R_A$, $R_{B7C6}$ a $R_8$, zjednodušíme ich na $R_{ekv}$ aby sme zistili celkový odpor:
    \[
        R_{ekv} = R_A + R_{B7C6} + R_8 = \frac{548250}{2371} + \frac{227973153900}{769752263} + 250 = 
        \frac{252383900}{324653} = \SI{777,3959}{\ohm}
    \]
\end{center}
\end{figure}
