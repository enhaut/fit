\begin{figure}[!ht]
\begin{center}
    Do rovníc si dosadíme rovnice prúdov:
    \[
        A: \frac{U - U_A}{R_1} - \frac{U_A}{R_2} - \frac{U_A - U_B}{R_3} = 0
    \]
    \[
        B: \frac{U_A - U_B}{R_3} + 0,7 - \frac{U_B - U_C}{R_5} = 0
    \]
    \[
        C: \frac{U_B - U_C}{R_5} - 0,7 + 0,8 - \frac{U_C}{R_4} = 0
    \]
    Z 3 rovníc o 3 neznámych vypočítame $U_A$, $U_B$ a $U_C$ napríklad pomocou Cramerovho pravidla alebo dosadzovacou metódou (ako som to spravil ja na pomocnom papieri (/riesenia/3/napatia.jpg) a už som to nestihol prepísať sem :().
    \[
        U_A = 68,6067 V
    \]
    \[
        U_B = 60,2811 V
    \]
    \[
        U_C = \frac{4612778}{257115} = 31,8406 V
    \]
    Keď už máme vypočítane $U_A$ môžeme pomocou neho a Ohmovho zákona vypočítať prúd $I_{R2}$:
    \[
        I_{R2} = \frac{U_A}{R2} =
        \frac{68,6067}{45} = 1,15246 A
    \]
    \[
        U_{R2} = U_A = 68,6067 V
    \]
\end{center}
\end{figure}
