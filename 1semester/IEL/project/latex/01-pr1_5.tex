\begin{figure}[!ht]
\begin{center}
    \begin{center}
\begin{circuitikz}[]
\draw
  (0.5,4) to[resistor, l_=$R_A$, i=$I$, -*] ++(2,0)
  (2.5, 3) to[short, *-*] ++(0, 2)
  to[resistor, l=$R_{B7}$, -*, i=$I_{RB7}$, v=$U_{RB7}$] ++(3, 0)
  to[short] ++(0,-4)
  (0.5,1) to[resistor, l_=$R_8$] ++(5,0)
  (2.5,3) to[resistor, l=$R_{C6}$, -*, i=$I_{RC6}$, v=$U_{RC6}$] ++(3,0)
  (0.5,3) to[short] ++(0,1)
  (0.5,1) to[short] ++(0,1)
  (0.5,3) to[dcvsource, v_>=$U_{12}$] ++(0,-1)
;\end{circuitikz}
\end{center}
    Keďže rezistory $R_{B7}$ a $R_{C6}$ sú zapojené paralerne je na nich rovnaké napätie, teda:
    \[
        U_{RB7} = U_{C6} = U_{RB7C6}
    \]
    Vypočítame prúd prechádzajúci odporom $R_{C6}$:
    \[
        I_{C6} = \frac{U_{RB7C6}}{R_{C6}} = 
        \frac{\frac{433148992410}{5984022269}}{\frac{2219300}{2371}} = 
        \frac{433144899241}{5601155892700} = 0,07733 A
    \]
    Kedže $R_C$ a $R_6$ sú zapojene sériovo, bude cez ne prechádzať rovnaký prúd:
    \[
        I_{R6} = I_C = I_{C6} = \frac{433144899241}{560115589270} = 0,7733 A
    \]
    Z prechádzajúceho prúdu vieme pomocou Ohmového zákona vypočítať napätie na odpore:
    \[
        U_{R6} = R_6 * I_{R6} = 
        800 * \frac{433144899241}{5601155892700} = 
        \frac{3465159193928}{56011558927} = 61,8651 V
    \]
\end{center}
\end{figure}

