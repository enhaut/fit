\begin{figure}[!ht]
\begin{center}
    \input{circuits/1/9.tex}
    Keďže rezistory $R_{B7}$ a $R_{C6}$ sú zapojené paralerne je na nich rovnaké napätie, teda:
    \[
        U_{RB7} = U_{C6} = U_{RB7C6}
    \]
    Vypočítame prúd prechádzajúci odporom $R_{C6}$:
    \[
        I_{C6} = \frac{U_{RB7C6}}{R_{C6}} = 
        \frac{\frac{433148992410}{5984022269}}{\frac{2219300}{2371}} = 
        \frac{433144899241}{5601155892700} = 0,07733 A
    \]
    Kedže $R_C$ a $R_6$ sú zapojene sériovo, bude cez ne prechádzať rovnaký prúd:
    \[
        I_{R6} = I_C = I_{C6} = \frac{433144899241}{560115589270} = 0,7733 A
    \]
    Z prechádzajúceho prúdu vieme pomocou Ohmového zákona vypočítať napätie na odpore:
    \[
        U_{R6} = R_6 * I_{R6} = 
        800 * \frac{433144899241}{5601155892700} = 
        \frac{3465159193928}{56011558927} = 61,8651 V
    \]
\end{center}
\end{figure}

