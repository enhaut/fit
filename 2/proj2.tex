%%% attributes specified in assignment
\documentclass[twocolumn, 11pt, a4paper]{article}
\usepackage[IL2]{fontenc}
\usepackage[left=1.5cm, top=2.5cm, text={18cm, 25cm}]{geometry}
\usepackage{times} %% font times
\usepackage{amssymb, amsmath, amsthm, dsfont}
%%%

%%% my attributes
\usepackage[utf8]{inputenc}
\usepackage[czech]{babel}
\usepackage[draft=false, hidelinks]{hyperref} % draft=false removes draft warning during compilation
%%%

\newtheorem{definition}{Definice}
\newtheorem{lemma}{Věta}


\begin{document}
	\begin{titlepage}
		\begin{center}
			\textsc{
				\Huge{Vysoké učení technické v~Brně}\\[0,4em]
				\huge{Fakulta informačních technologií}
			}
		\end{center}
		\begin{center}
			\vspace{
   				\stretch{0.370}
			}
			\LARGE{
				Typografie a publikování\,--\,2. projekt\\
				\hspace{0.1em} Sazba dokumentů a matematických výrazů}
			\vspace{
				\stretch{0.63}
			}			
		\end{center}
    	{\Large 2022 \hfill Samuel Dobroň (xdobro23)}
	\end{titlepage}

\section*{Úvod}
	V~této úloze si vyzkoušíme sazbu titulní strany, ma\-te\-ma\-tic\-kých vzorců, prostředí a dalších textových struktur obvyklých pro technicky zaměřené texty (například rovnice (\ref{eq2}) nebo Definice \ref{def2} na straně \pageref{def2}). Pro vytvoření těchto odkazů používáme příkazy \verb|\label|, \verb|\ref| a \verb|\pageref|.
	
	Na titulní straně je využito sázení nadpisu podle optického středu s~využitím zlatého řezu. Tento postup byl probírán na přednášce. Dále je na titulní straně použito odřádkování se zadanou relativní velikostí 0,4\,em a 0,3\,em.

\section{Matematický text}
	Nejprve se podíváme na sázení matematických symbolů a~výrazů v~plynulém textu včetně sazby definic a vět s~využitím balíku \texttt{amsthm}. Rovněž použijeme poznámku pod čarou s~použitím příkazu \verb|\footnote|. Někdy je vhodné použít konstrukci \verb|${}$| nebo \verb|\mbox{}|, která říká, že (matematický) text nemá být zalomen. 

	\begin{definition}
	\emph{Nedeterministický Turingův stroj} (NTS) je šes\-ti\-ce tvaru
		$M = (\mathnormal{Q}, \Sigma, \Gamma, \delta, q_{0}, q_{F})$, kde:
		\begin{itemize}
			\item{
				$Q$ je konečná množina \textnormal{vnitřních (řídicích) stavů},
			}
			\item{
				$\Sigma$ je konečná množina symbolů nazývaná \textnormal{vstupní abeceda}, $\Delta \not\in \Sigma$,
			}
			\item{
				$\Gamma$ je konečná množina symbolů, $\Sigma \subset \Gamma$, $\Delta \in \Gamma$, nazývaná \textnormal{pásková abeceda},
			}
			\item{
				$\delta: (Q \setminus \{q_F\} ) \times \Gamma \rightarrow 2^{Q \times (\Gamma \cup \{L, R\})}$, kde $L, R \notin \Gamma$, je parciální \textnormal{přechodová funkce}, a
			}
			\item{
				$q_0 \in Q$ je \textnormal{počáteční stav} a $q_f \in Q$ je \textnormal{koncový stav}.
			}
		\end{itemize}
	\end{definition} 

	Symbol $\Delta$ značí tzv. \textit{blank} (prázdný symbol), který se vyskytuje na místech pásky, která nebyla ještě použita.

	\textit{Konfigurace pásky} se skládá z~nekonečného řetězce, který reprezentuje obsah pásky, a pozice hlavy na tomto řetězci. Jedná se o~prvek množiny
		$\{\gamma \Delta^{\omega}\;|\;\gamma \in \Gamma^*\} \times \mathbb{N}$\footnote{
		Pro libovolnou abecedu $\Sigma$ je $\Sigma^{\omega}$ množina všech \textit{nekonečných} řetězců nad $\Sigma$, tj. nekonečných posloupností symbolů ze $\Sigma$.
		}.
	\textit{Konfiguraci pásky} obvykle zapisujeme jako
		$\Delta xyz\underline{z}x\Delta $... (podtržení značí pozici hlavy).
	\textit{Konfigurace stroje} je pak dána stavem řízení a konfigurací pásky. Formálně se jedná o~prvek množiny
		$Q \times \{\gamma \Delta^{\omega}\;|\;\gamma \in \Gamma^*\} \times \mathbb{N}$.

	\subsection{Podsekce obsahující definici a větu}
		\begin{definition}
		\label{def2}
			\textnormal{Řetězec $w$ nad abecedou $\Sigma$ je přijat NTS}~$M$,
			\textit{
				jestliže $M$ při aktivaci z~počáteční konfigurace pásky
				$\underline{\Delta}w\Delta ...$ a počátečního stavu $q_0$ může zastavit přechodem do koncového stavu $q_F$, tj.
				$(q_0, \Delta w\Delta^{\omega}, 0) \overset{*}{\underset{M}{\vdash}} (q_F, \gamma, n)$
				pro nějaké
				$\gamma \in \Gamma^*$ a $n \in \mathbb{N}$.
			}
	
			\textit{Množinu} $L(M)\, =\, \{w\ \,|\ \,w$\!  je\;přijat\;NTS\;$M\} \subseteq \Sigma^*$ nazýváme \textnormal{jazyk přijímaný NTS} $M$.	
		\end{definition}
		Nyní si vyzkoušíme sazbu vět a důkazů opět s~použitím balíku \texttt{amsthm}.
		\begin{lemma}Třída jazyků, které jsou přijímány NTS, odpovídá \textnormal{rekurzivně vyčíslitelným jazykům.}
		\end{lemma}

\section{Rovnice}
	Složitější matematické formulace sázíme mimo plynulý text. Lze umístit několik výrazů na jeden řádek, ale pak je třeba tyto vhodně oddělit, například příkazem \verb|\quad|.
	
$$
		x^2 - \sqrt[4]{y_1 \ast y^3_{2}} \quad x > y_1 \geq y_2 \quad z_{z_z} \not= \alpha^{\alpha^{\alpha_3}_2}_1
$$

	V~rovnici (\ref{eq1}) jsou využity tři typy závorek s~různou explicitně definovanou velikostí.
	
	\begin{eqnarray}
		x &=& \bigg\{a \oplus \Big[b \cdot \big(c \ominus d \big)\Big] \bigg\}^{4/2} \label{eq1}\\
		y &=& \lim_{\beta\to\infty} \frac{\tan^2 \beta \minus \sin^3 \beta}{\frac{1}{\frac{1}{\log_{42}x} + \frac{1}{2}}}\label{eq2}
	\end{eqnarray}
	
	V~této větě vidíme, jak vypadá implicitní vysázení limity
	$\lim_{n\to\infty} f(n)$
	v~normálním odstavci textu. Podobně je to i s~dalšími symboly jako
	$\bigcup_{N \in \mathcal{M}} N$ či $\sum_{j=0}^n x^2_j$.
	S~vynucením méně úsporné sazby příkazem \verb|\limits| budou vzorce vysázeny v~podobě
	$\lim\limits_{n\to\infty} f(n)$ a $\sum\limits^n_{j=0} x^2_j$. 

\section{Matice}
Pro sázení matic se velmi často používá prostředí \texttt{array} a závorky (\verb|\left|, \verb|\right|). 

$$
	\textbf{\large A} =
	\left|
		\begin{array}{cccc}
			a_{11} & a_{12} & \dots & a_{1n} \\
			a_{21} & a_{22} & \dots & a_{2n} \\
			\vdots & \vdots & \ddots &\vdots \\
			a_{m1} & a_{m2} & \dots & a_{mn}
		\end{array}
	\right|
	=
	\left|
		\begin{array}{cc}
			t & u \\
			v & w
		\end{array}
	\right|
	=
	tw \minus uv
$$

Prostředí \texttt{array} lze úspěšně využít i jinde.

$$
	\binom{n}{k}
	=
	\Bigg\{
		\begin{array}{c l}
			\frac{n!}{k!( n\minus k)!}	& \text{pro } 0 \leq k \leq n \\
			0							& \text{pro } k > n \text{ nebo } k < 0 
		\end{array}
$$
\end{document}
